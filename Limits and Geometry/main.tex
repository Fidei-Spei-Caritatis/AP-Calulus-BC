\documentclass[12pt]{article}
\usepackage[paper=legalpaper,margin=2cm]{geometry}

\usepackage{mathtools, amssymb, amsthm}
\usepackage{xcolor}
\usepackage{ragged2e}
\usepackage{graphicx}
\usepackage{fancyhdr}
\usepackage{tikz}
\usepackage{pgfplots}
\usetikzlibrary{arrows}
\usepackage{empheq}
\usepackage[most]{tcolorbox}
\everymath{\displaystyle}

\usepackage{hyperref}
\hypersetup{
    colorlinks=true,
    linkcolor=blue,
    filecolor=magenta,      
    urlcolor=cyan,
    pdftitle={Overleaf Example},
    pdfpagemode=FullScreen,
    }

\pagestyle{fancy}
\fancyhf{}
\rhead{\small {© MMXXII All Rights Reserved, Aiden Rosenberg}}

\title{Limits and Geometry}
\author{Aiden Rosenberg}
\date{December MMXXII A.D}
\newtheorem{prop}{Proposition}

\begin{document}

\maketitle
\thispagestyle{fancy}

\section{Problem}
Let $P(a,a^2)$ be a point on the parabola $y=x^2$, $a>0$. Let $O$ denote the origin and $(0,b)$ the $y$-intercept of the perpendicular bisector of the line segment $\overline{OP}$. Find the $\lim_{P\to O} b$.
\section{Conjecture}
\begin{prop}
The $\lim_{P\to O} b=k$ such that $k\in\mathbb{R^{+}}$.
\end{prop}
\section{Graph}
\begin{center}
\begin{tikzpicture}
\begin{axis}[
    axis lines = center,
    axis line style={->},
    xlabel = \(x\),
    ylabel = {\(f(x)\)},
    ticks=none, 
    xmin=-1,ymin=-1,ymax=2,xmax=2,
    every axis plot/.append style={thick},
   legend pos=outer north east
    ]
\addplot [
    domain=-5:5, 
      range=-1:1,
      smooth,
      samples=50, 
    color=violet,
]
{x^2};
\addlegendentry{\(f(x)=x^2\)}
%Here the blue parabola is defined
\addplot[
    domain=-5:5, 
    range=0:1
    samples=50, 
    color=purple
]
{(-1)*(x-0.5)+0.5};
\addlegendentry{Perpendicular Bisector}

\addplot [
    domain=0:1, 
    yrange=0:1
    xmin = 0, xmax=1
    samples=50, 
    color=teal,
]
{x};
\addlegendentry{Secant Line}
\addplot [color=magenta,mark=*]
    coordinates {
    (1,1)
    };
\addlegendentry{$P$}
\addplot [color=cyan,mark=*]
    coordinates {
    (0,1)
    };
\addlegendentry{$(0,b)$}
\addplot [fill=white,mark=*]
    coordinates {
    (0,0)
    };
\addlegendentry{$O$}
\end{axis}
\end{tikzpicture}
\end{center}

\section{Analysis}

\subsection{Numerical Analysis}
\begin{table}[h]
\centering
\begin{tabular}{l||l|l|l|l|l}
$a$ & 0.1 & 0.01 & 0.001 & 0.0001 & 0.00001 \\ \hline
$b$ &  0.505   &  0.50005   &    0.5000005   & 0.500000005 &0.50000000005     \\
\end{tabular}
\end{table}
The table above represents simulated values for $b$ using graphical analysis via the Desmos engine. Link to the interactive graph: \url{https://www.desmos.com/calculator/kwepdntlyx} 
\subsection{Algebraic Analysis}
Let $m$ denote the slope of the secant line of the point $(0,0)$ and $(0,a)$ $\therefore m=\frac{f(a)}{a} = \frac{a^2}{a}=a$. The equation for the perpendicular bisector of the line can be written as $y_{\bot}=\frac{-1}{a}(x-x_m)+y_m$ where $(x_m, y_m)$ is the midpoint of the secant line. The point $(x_m, y_m)$ can be expressed as $\biggr(\frac{a}{2}, \frac{f(a)}{2}\biggr) \Longrightarrow y_{\bot}=\frac{-1}{a}\biggr(x-\frac{a}{2}\biggr)+\frac{f(a)}{2} \xRightarrow[]{\text{Simplifying}} y_{\bot}=\frac{-x}{a}+ \underbrace{\frac{1}{2}+ \frac{a^2}{2}}_{\text{Real Number}}$. When $x=0 \Longrightarrow y=\underbrace{\frac{1+a^2}{2}}_{y\text{-intercept}}=b$. The $\lim_{P\to O} b = \lim_{a\to 0} \frac{1+a^2}{2} =\tcbhighmath{\frac{1}{2}}$. 

\begin{FlushRight}
\textbf{Q.E.D}.
\end{FlushRight}



\end{document}
