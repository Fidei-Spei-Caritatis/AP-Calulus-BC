\documentclass[12pt]{article}
\usepackage[paper=A4,margin=2cm]{geometry}

\usepackage {mathtools, amssymb, amsthm}
\usepackage{enumerate}
\usepackage{enumitem}
\usepackage{fancyhdr}
\usepackage{tabularx}
\usepackage{graphicx}

\pagestyle{fancy}
\fancyhf{}
\rhead{\small {© 2022 All Rights Reserved, Aiden Rosenberg}}
\rfoot{Page \thepage}

\setlength{\droptitle}{-6em}
\everymath{\displaystyle}

\title{Improper Integral Practice
}
\author{Aiden Rosenberg}
\date{November 28, 2022 A.D}

\begin{document}
\maketitle
    \section*{Question 1}
\begin{enumerate}[label=\Alph*.]
	\item \(\int \frac{dx}{x^2+1} = \arctan x + C \)
	\item The \(\int_{0}^{\infty} \frac{dx}{1+x^2}\) is improper because of an indefinite integral i.e for \( x\in[0,\infty) \).
		\item \(\int_{0}^{\infty} \frac{dx}{1+x^2}  =\underbrace{\int_{0}^{1} \frac{dx}{1+x^2}}_{\text{Definite Integral}} + \underbrace{\int_{1}^{\infty} \frac{dx}{1+x^2}}_{\text{Improper Integral}}\). Since \(\frac{1}{1+x^2} < \frac{1}{x^2}\) on $(1,\infty)$ and $\int_{1}^{\infty} \frac{dx}{x^2}$ converges. Therefore by the comparison test $\int_{1}^{\infty} \frac{dx}{1+x^2}$ converges. 
		\item $\int_{0}^{\infty} \frac{dx}{1+x^2}  =\int_{0}^{1} \frac{dx}{1+x^2} + \int_{1}^{\infty} \frac{dx}{1+x^2} = \arctan x \biggr\rvert_{0}^{1} + \lim_{b \to \infty} \arctan x \biggr\rvert_{1}^{b} = \boxed{\frac{\pi}{2}}$
		\end{enumerate}
    \section*{Question 2}
		\begin{enumerate}
			\item $\int x^{-3/5} \, dx  = \frac{5x^{2/5}}{2} + C $
			\item The $\int_{0}^{\infty} x^{-3/5} \, dx $ is improper because of an indefinite integral i.e for \( x\in(0,\infty) \). 
			\item \(\int_{0}^{\infty} \frac{dx}{1+x^2}  =\underbrace{\int_{0}^{1} x^{-3/5} \, dx }_{\text{Converges}} + \underbrace{\int_{1}^{\infty} x^{-3/5} \, dx}_{\text{Diverges}}\). Since $\int_{1}^{\infty }\frac{1}{x^P} \, dx$ converges when $P > 1$ and diverges when $P \leq 1$, and $\frac{3}{5} < 1$ therefore the $\int_{1}^{\infty} \frac{dx}{x^{3/5}}$ diverges. 
			\item $\int_{0}^{1} x^{-3/5} = \lim_{a\to 0} \frac{5x^{2/5}}{2} \biggr\rvert_{a}^{1} = \boxed{\frac{5}{2}}$
		\end{enumerate}
    \section*{Question 3}
		\begin{enumerate}
			\item  $\int x^{-1/3} \, dx  = \frac{3x^{2/3}}{2} + C $
			\item The $\int_{-8}^{1} x^{-1/3} \, dx $ is improper because the $\lim_{x\to 0^+} x^{-1/3} = \infty$ and the$\lim_{x\to 0^-} x^{-1/3} = -\infty$  therefore there is a vertical asymptote at $x=0$.
			\item \(\int_{-8}^{1} x^{-1/3} \, dx  =\underbrace{\int_{-8}^{0} x^{-1/3} \, dx }_{\text{Converges}} + \underbrace{\int_{0}^{1} x^{-1/3} \, dx}_{\text{Converges}}\). Since $\int_{0}^{1}\frac{1}{x^P} \, dx$ converges when $P < 1$ and diverges when $P \geq 1$, and $\frac{1}{3} < 1$ therefore the $\int_{0}^{1} \frac{dx}{x^{1/3}}$ and $\int_{-1}^{0} x^{-1/3} \, dx $ converges. 
			\item $\int_{-8}^{1} x^{-1/3} \, dx = \lim_{a\to 0^-} \frac{3x^{2/3}}{2} \biggr\rvert_{-8}^{a} + \lim_{b\to 0^+} \frac{3x^{2/3}}{2} \biggr\rvert_{b}^{1} = \boxed{-\frac{9}{2}}$
		\end{enumerate}
    \section*{Question 4}
		\begin{enumerate}
			\item $\int \frac{e^{1/x}}{x^2} \, dx = -e^{1/x} + C$
			\item The $\int_{0}^{\ln 2} \frac{e^{1/x}}{x^2} \, dx $ is improper because the $\lim_{x\to 0^+} \frac{e^{1/x}}{x^2} = \infty$  therefore there is a vertical asymptote at $x=0$.
			\item The $\underbrace{\int_{0}^{\ln 2} \frac{e^{1/x}}{x^2} \, dx}_{\text{Converges}}$. Since \(\frac{e^{1/x}}{x^2} > \frac{1}{x^2}\) on $(0,\ln 2)$ and $\int_{0}^{\ln 2} \frac{dx}{x^2}$ diverges. Therefore by the comparison test $\int_{0}^{\ln 2} \frac{e^{1/x}}{x^2}$ diverges.  
		\end{enumerate}
    \section*{Question 5}
		\begin{enumerate}
			\item $\int \frac{dx}{x^2+5x+6} =\int \frac{dx}{(x+2)(x+3)} =\ln\bigg|\frac{x+2}{x+3}\bigg| + C$
			\item The $\int_{-1}^{\infty} \frac{dx}{x^2+5x+6}$ is improper because of an indefinite integral i.e for $x \in[-1, \infty)$.
				\item The $\int_{-1}^{\infty}\underbrace{\int_{-1}^{1} \frac{dx}{x^2+5x+6}}_{\text{Definite integral}} + \underbrace{\int_{1}^{0} \frac{dx}{x^2+5x+6}}_{\text{Improper integral}}$. Since $\int_{-1}^{1} \frac{dx}{x^2+5x+6}$ converges and $\frac{1}{x^2+5x+6}<\frac{1}{x^2}$ for $x\in(1,\infty)$ and $\int_{1}^{\infty} \frac{dx}{x^2}$ converges, therefore by the comparison test $\int_{1}^{\infty} \frac{dx}{x^2+5x+6}$ converges.
				\item $\int_{-1}^{\infty} \frac{dx}{x^2+5x+6} = \lim_{b\to\infty} \ln\bigg|\frac{x+2}{x+3}\bigg|\biggr\rvert_{-1}^{b} = \lim_{b\to\infty}\underbrace{\ln\bigg|\frac{b+2}{b+3}\bigg|}_{\ln (1)} - \ln \frac{1}{2}= \ln 2$
				\end{enumerate}
    \section*{Question 6}
				\begin{enumerate}
					\item $\int \tan x \, dx =\ln|\sec x| + C$
					\item The $\int_{0}^{\frac{\pi}{2}} \tan x \, dx$ is improper because the $\lim_{x\to\frac{\pi}{2}^-} \tan x = \infty$, therefore there is a vertical asymptote at $x = \frac{\pi}{2}$.
				\item $\underbrace{\int_{0}^{\frac{\pi}{2}} \tan x \, dx}_{\text{Diverges}}$. Since $\lim_{x\to\frac{\pi}{2}} \tan x = \frac{1}{\frac{\pi}{2}-x}$ and since the $\int_{0}^{\frac{\pi}{2}} \frac{1}{\frac{\pi}{2}-x}$ diverges then  $\int_{0}^{\frac{\pi}{2}} \tan x \, dx$ diverges.  
                    \item $\int_{0}^{\frac{\pi}{2}} \tan x \, dx = \lim_{b\to \frac{\pi}{2}} \ln|\sec x| \biggr\rvert_{0}^{b} = \lim_{b\to \frac{\pi}{2}} \ln|\sec b| - \ln(1) = \infty $
				\end{enumerate}
    \section*{Question 7}
    \begin{enumerate}
        \item $\int \frac{dx}{\sqrt{x-1}} = 2\sqrt{x-1} +C$
        \item The $\int_{5}^{\infty} \frac{dx}{\sqrt{x-1}}$ is improper because of an indefinite integral i.e for \( x\in[5,\infty) \).
        \item $\underbrace{\int_{5}^{\infty} \frac{dx}{\sqrt{x-1}}}_{\text{Diverges}}$. Since $\frac{1}{\sqrt{x-1}} > \frac{1}{x^{1/2}}$ on $(5,\infty)$ and $\int_{5}^{\infty} x^{-1/2} \, dx$ diverges, therefore by the comparison test $\int_{5}^{\infty} \frac{dx}{\sqrt{x-1}}$ diverges.
        \item $\int_{5}^{\infty} \frac{dx}{\sqrt{x-1}} = \lim_{b\to\infty} 2\sqrt{x-1}\biggr\rvert_{5}^{b} = \infty$
    \end{enumerate}
    \section*{Question 8}
    \begin{enumerate}
        \item $\int \frac{dx}{\sqrt{4-x}}=-2\sqrt{4-x}+C$
        \item The $\int_{0}^{4} \frac{dx}{\sqrt{4-x}}$ is improper because the $\lim_{x\to4^-} \frac{dx}{\sqrt{4-x}} = \infty$, therefore there is a vertical asymptote at $x=4$.
        \item $\int_{0}^{4} \frac{dx}{\sqrt{4-x}}= \int_{0}^{4} \frac{dx}{\sqrt{x}} = \underbrace{\int_{0}^{1} \frac{dx}{\sqrt{x}}}_{\text{Converges}} + \underbrace{\int_{1}^{4} \frac{dx}{\sqrt{x}}}_{\text{Definite integral}}$. Since $\int_{0}^{1}\frac{1}{x^P} \, dx$ converges when $P < 1$ and diverges when $P \geq 1$, and $\frac{1}{2} < 1$ therefore the $\int_{0}^{1} \frac{dx}{\sqrt{x}}$ diverges. 
        \item $\int_{0}^{4} \frac{dx}{\sqrt{4-x}}=\lim_{b\to\infty}-2\sqrt{x-1}\biggr\rvert_{0}^{4} = \infty$
    \end{enumerate}
\section*{Question 9}
\begin{enumerate}
    \item $\int \frac{dx}{x^2+5x+6} =\int \frac{dx}{(x+2)(x+3)} =\ln\bigg|\frac{x+2}{x+3}\bigg| + C$
    \item The $\int_{-5}^{0} \frac{dx}{x^2+5x+6}$ is improper because the $\underbrace{\lim_{x\to-2^-} \frac{1}{x^2+5x+6}}_{-\infty}  \neq \underbrace{\lim_{x\to-2^+} \frac{1}{x^2+5x+6}}_{\infty}$ and $\underbrace{\lim_{x\to-3^-} \frac{1}{x^2+5x+6}}_{\infty}  \neq \underbrace{\lim_{x\to-3^+} \frac{1}{x^2+5x+6}}_{-\infty}$, therefore there is a vertical asymptotes at $x=-2$ and $x=-3$.
    \item $\int_{-5}^{0} \frac{dx}{x^2+5x+6} = \underbrace{\int_{-5}^{-3} \frac{dx}{x^2+5x+6}}_{\text{Diverges}} + \underbrace{\int_{-3}^{2} \frac{dx}{x^2+5x+6}}_{\text{Diverges}} + \underbrace{\int_{-2}^{0}\frac{dx}{x^2+5x+6}}_{\text{Diverges}}$. Since $\int_{-2}^{0}\frac{1}{x^2+5x+6} =\lim_{a\to-2} \ln\bigg|\frac{x+2}{x+3}\bigg| \biggr\rvert_{a}^{0}= 2\ln 3 - \underbrace{\ln (0)}_{\infty}  = \infty$ then $\int_{-5}^{0} \frac{dx}{x^2+5x+6}$ diverges.
\end{enumerate}
\section*{Question 10}
\begin{enumerate}
     \item $\int \frac{dx}{\sqrt{x-1}} = 2\sqrt{x-1} +C$
     \item The $\int_{1}^{5} \frac{dx}{\sqrt{x-1}}$ is improper because $\lim_{x\to 1^+} \frac{1}{\sqrt{x-1}} = \infty$ and therefore there is a vertical asymptote at $x=1$.
     \item $\int_{1}^{5} \frac{dx}{\sqrt{x}} = \underbrace{\int_{0}^{1} \frac{dx}{\sqrt{x}}}_{\text{Converges}} + \underbrace{\int_{1}^{5} \frac{dx}{\sqrt{x}}}_{\text{Definite integral}}$. Since $\int_{0}^{1} \frac{1}{x^P} \, dx$ converges when $P<1$ and diverges when $P\geq 1$ and $\frac{1}{2}<1$, therefore the $\int_{0}^{1} \frac{dx}{\sqrt{x}}$ converges.
     \item $\int_{1}^{5} \frac{dx}{\sqrt{x}} = \lim_{a\to 0} 2\sqrt{x-1}\biggr\rvert_{a}^{1} = 4$
     
\end{enumerate}
\end{document}